\documentclass[12pt,]{article}
\usepackage{lmodern}
\usepackage{amssymb,amsmath}
\usepackage{ifxetex,ifluatex}
\usepackage{fixltx2e} % provides \textsubscript
\ifnum 0\ifxetex 1\fi\ifluatex 1\fi=0 % if pdftex
  \usepackage[T1]{fontenc}
  \usepackage[utf8]{inputenc}
\else % if luatex or xelatex
  \ifxetex
    \usepackage{mathspec}
  \else
    \usepackage{fontspec}
  \fi
  \defaultfontfeatures{Ligatures=TeX,Scale=MatchLowercase}
    \setmainfont[]{Helvetica}
\fi
% use upquote if available, for straight quotes in verbatim environments
\IfFileExists{upquote.sty}{\usepackage{upquote}}{}
% use microtype if available
\IfFileExists{microtype.sty}{%
\usepackage{microtype}
\UseMicrotypeSet[protrusion]{basicmath} % disable protrusion for tt fonts
}{}
\usepackage{hyperref}
\hypersetup{unicode=true,
            pdftitle={Diagnostic Potential \& the Interactive Dynamics of the Colorectal Cancer Virome},
            pdfborder={0 0 0},
            breaklinks=true}
\urlstyle{same}  % don't use monospace font for urls
\usepackage{graphicx,grffile}
\makeatletter
\def\maxwidth{\ifdim\Gin@nat@width>\linewidth\linewidth\else\Gin@nat@width\fi}
\def\maxheight{\ifdim\Gin@nat@height>\textheight\textheight\else\Gin@nat@height\fi}
\makeatother
% Scale images if necessary, so that they will not overflow the page
% margins by default, and it is still possible to overwrite the defaults
% using explicit options in \includegraphics[width, height, ...]{}
\setkeys{Gin}{width=\maxwidth,height=\maxheight,keepaspectratio}
\IfFileExists{parskip.sty}{%
\usepackage{parskip}
}{% else
\setlength{\parindent}{0pt}
\setlength{\parskip}{6pt plus 2pt minus 1pt}
}
\setlength{\emergencystretch}{3em}  % prevent overfull lines
\providecommand{\tightlist}{%
  \setlength{\itemsep}{0pt}\setlength{\parskip}{0pt}}
\setcounter{secnumdepth}{0}
% Redefines (sub)paragraphs to behave more like sections
\ifx\paragraph\undefined\else
\let\oldparagraph\paragraph
\renewcommand{\paragraph}[1]{\oldparagraph{#1}\mbox{}}
\fi
\ifx\subparagraph\undefined\else
\let\oldsubparagraph\subparagraph
\renewcommand{\subparagraph}[1]{\oldsubparagraph{#1}\mbox{}}
\fi
\usepackage{setspace}
\doublespacing
\usepackage[vmargin=1in,hmargin=1in]{geometry}
\usepackage{lineno}
\linenumbers
\usepackage{parskip}
\setlength{\parskip}{7.5pt}
\exhyphenpenalty=10000
\hyphenpenalty=10000

\newcommand{\beginsupplement}{%
        \setcounter{table}{0}
        \renewcommand{\thetable}{S\arabic{table}}%
        \setcounter{figure}{0}
        \renewcommand{\thefigure}{S\arabic{figure}}%
     }

  \title{Diagnostic Potential \& the Interactive Dynamics of the Colorectal
Cancer Virome}
    \usepackage{authblk}
    \font\myfont=Helvetica at 12pt
    \font\nextfont=Helvetica at 10pt
                        \author[1]{\myfont Geoffrey D Hannigan}
                    \author[2]{\myfont Melissa B Duhaime}
                    \author[3]{\myfont Mack T Ruffin IV}
                    \author[1]{\myfont Charlie C Koumpouras}
                    \author[1,*]{\myfont Patrick D Schloss}
                            \affil[1]{\nextfont Department of Microbiology \& Immunology, University of Michigan, Ann
Arbor, Michigan, 48109}
                    \affil[2]{\nextfont Department of Ecology and Evolutionary Biology, University of Michigan,
Ann Arbor, Michigan, 48109}
                    \affil[3]{\nextfont Department of Family and Community Medicine, Pennsylvania State
University Hershey Medical Center, Hershey, Pennsylvania, 17033}
                    \affil[*]{\nextfont To whom correspondence may be addressed.}
            \date{}

\begin{document}
\maketitle

~

\textbf{\emph{Corresponding Author Information}}\\
Patrick D Schloss, PhD\\
1150 W Medical Center Dr.~1526 MSRB I\\
Ann Arbor, Michigan 48109\\
Phone: (734) 647-5801\\
Email: pschloss@umich.edu

\textbf{Article Class}: Research Article

\newpage

\section{Abstract}\label{abstract}

Viruses are associated with many human cancers, largely due to their
mutagenic and functionally manipulative abilities. Despite this, cancer
microbiome studies have almost exclusively focused on bacteria instead
of viruses. We began evaluating the cancer virome by focusing on
colorectal cancer, a primary cause of morbidity and mortality throughout
the world, and a cancer linked to altered colonic bacterial community
compositions while the virome role remains unknown. We used 16S rRNA
gene, whole shotgun metagenomic, and purified virus metagenomic
sequencing of stool to evaluate the differences in human colorectal
cancer virus and bacterial community composition. Through random forest
modeling we identified differences in the healthy and colorectal cancer
virome. The cancer-associated virome consisted primarily of temperate
bacteriophages that were also predicted to be bacteria-virus community
network hubs. These results provide foundational evidence that
bacteriophage communities are associated with colorectal cancer and
likely impact cancer progression by altering the bacterial host
communities.

\section{Importance}\label{importance}

Colorectal cancer is a leading cause of cancer-related death in the
United States and worldwide. Its risk and severity have been linked to
colonic bacterial community composition. Although viruses have been
linked to other cancers and diseases, little is known about colorectal
cancer virus communities. We addressed this knowledge gap by identifying
differences in colonic virus communities in the stool of colorectal
cancer patients and how they compared to bacterial community
differences. The results suggested an indirect role for the virome in
impacting colorectal cancer by modulating their associated bacterial
community. These findings both support a biological role for viruses in
colorectal cancer and provide a new understanding of basic colorectal
cancer etiology.

\newpage

\section{Introduction}\label{introduction}

Due to their mutagenic abilities and propensity for functional
manipulation, human viruses are strongly associated with, and in many
cases cause, cancer (1--4). Because bacteriophages (i.e.~viruses that
specifically infect bacteria) are crucial for bacterial community
stability and composition (5--7) and have been implicated as oncogenic
agents (8--11), bacteriophages have the potential to indirectly impact
cancer. The gut virome (i.e.~the virus community of the gut) therefore
has the potential to impact health and disease. Altered human virome
composition and diversity have been identified in diseases including
periodontal disease (12), HIV (13), cystic fibrosis (14), antibiotic
exposure (15, 16), urinary tract infections (17), and inflammatory bowel
disease (18). The strong association of bacterial communities with
colorectal cancer and the precedent for the virome to impact other human
diseases suggest that colorectal cancer may be associated with altered
virus communities.

Colorectal cancer is the second leading cause of cancer-related deaths
in the United States (19). The US National Cancer Institute estimates
over 1.5 million Americans were diagnosed with colorectal cancer in 2016
and over 500,000 Americans died from the disease (19). Growing evidence
suggests that an important component of colorectal cancer etiology may
be perturbations in the colonic bacterial community (8, 10, 11, 20, 21).
Work in this area has led to a proposed disease model in which bacteria
colonize the colon, develop biofilms, promote inflammation, and enter an
oncogenic synergy with the cancerous human cells (22). This association
also has allowed researchers to leverage bacterial community signatures
as biomarkers to provide accurate, noninvasive colorectal cancer
detection from stool (8, 23, 24). While an understanding of colorectal
cancer bacterial communities has proven fruitful both for disease
classification and for identifying the underlying disease etiology,
bacteria are only a subset of the colon microbiome. Viruses are another
important component of the colon microbial community that have yet to be
studied in the context of colorectal cancer. We evaluated disruptions in
virus and bacterial community composition in a human cohort whose stool
was sampled at the three relevant stages of cancer development: healthy,
adenomatous, and cancerous.

Colorectal cancer progresses in a stepwise process that begins when
healthy tissue develops into a precancerous polyp (i.e., adenoma) in the
large intestine (25). If not removed, the adenoma may develop into a
cancerous lesion that can invade and metastasize, leading to severe
illness and death. Progression to cancer can be prevented when
precancerous adenomas are detected and removed during routine screening
(26, 27). Survival for colorectal cancer patients may exceed 90\% when
the lesions are detected early and removed (26). Thus, work that aims to
facilitate early detection and prevention of progression beyond early
cancer stages has great potential to inform therapeutic development.

Here we address the knowledge gap of whether virus community composition
is altered in colorectal cancer and, if it is, how those differences
might impact cancer progression and severity. We also aimed to evaluate
the virome's potential for use as a diagnostic biomarker. The
implications of this study are threefold. \emph{First}, this work
supports a biological role for the virome in colorectal cancer
development and suggests that more than the bacterial members of the
associated microbial communities are involved in the process.
\emph{Second}, we present a supplementary, or even alternative,
virus-based approach for classification modeling of colorectal cancer
using stool samples. \emph{Third}, we provide initial support for the
importance of studying the virome as a component of the microbiome
ecological network, especially in cancer.

\section{Results}\label{results}

\subsection{Sample Collection and
Processing}\label{sample-collection-and-processing}

Our study cohort consisted of 90 human subjects, 30 of whom had healthy
colons, 30 of whom had adenomas, and 30 of whom had carcinomas
\textbf{(Figure \ref{sampleproc})}. Half of each stool sample was used
to sequence the bacterial communities using both 16S rRNA gene and
shotgun sequencing techniques. The 16S rRNA gene sequencing was
performed for a previous study, and the sequences were re-analyzed using
contemporary methods (8). The other half of each stool sample was
purified for virus like particles (VLPs) before genomic DNA extraction
and shotgun metagenomic sequencing. In the VLP purification, cells were
disrupted and extracellular DNA degraded \textbf{(Figure
\ref{sampleproc})} to allow the exclusive analysis of viral DNA within
virus capsids. In this manner, the \emph{extracellular virome} of
encapsulated viruses was targeted.

Each extraction was performed with a blank buffer control to detect
contaminants from reagents or other unintentional sources. Only one of
the nine controls contained detectable DNA at a minimal concentration of
0.011 ng/µl, thus providing evidence of the enrichment and purification
of VLP genomic DNA over potential contaminants \textbf{(Figure
\ref{qualcontrol} A)}. As expected, these controls yielded few sequences
and were almost entirely removed while rarefying the datasets to a
common number of sequences \textbf{(Figure \ref{qualcontrol} B)}. The
high quality phage and bacterial sequences were assembled into highly
covered contigs longer than 1 kb \textbf{(Figure \ref{contigqc})}.
Because contigs represent genome fragments, we further clustered related
bacterial contigs into operational genomic units (OGUs) and viral
contigs into operational viral units (OVUs) \textbf{(Figure
\ref{contigqc} - \ref{clustercontigqc})} to approximate organismal
units.

\subsection{Unaltered Diversity in Colorectal
Cancer}\label{unaltered-diversity-in-colorectal-cancer}

Microbiome and disease associations are often described as being of an
altered diversity (i.e., ``dysbiotic''). Therefore, we first evaluated
the influence of colorectal cancer on virome OVU diversity. We evaluated
differences in communities between disease states using the Shannon
diversity, richness, and Bray-Curtis metrics. We observed no significant
alterations in either Shannon diversity or richness in the diseased
states as compared to the healthy state \textbf{(Figure \ref{betaogu}
C-D)}. There was no statistically significant clustering of the disease
groups (ANOSIM p-value = 0.4, \textbf{Figure \ref{betaogu}}). Notably,
there was a significant difference between the few blank controls that
remained after rarefying the data and the other study groups (ANOSIM
p-value \textless{} 0.001, \textbf{Figure \ref{betaogunegative})},
further supporting the quality of the sample set. In summary, standard
alpha and beta diversity metrics were insufficient for capturing virus
community differences between disease states \textbf{(Figure
\ref{betaogu})}. This is consistent with what has been observed when the
same metrics were applied to 16S rRNA gene sequences and metagenomic
samples (8, 23, 24) and points to the need for alternate approaches to
detect the impact of colorectal cancer disease state on these
communities.

\subsection{Virome Composition in Colorectal
Cancer}\label{virome-composition-in-colorectal-cancer}

As opposed to the diversity metrics discussed above, OTU-based relative
abundance profiles generated from 16S rRNA gene sequences are effective
for classifying stool samples as originating from individuals with
healthy, adenomatous, or cancerous colons (8, 23). The exceptional
performance of bacteria in these classification models supports a role
for bacteria in colorectal cancer. We built off of these findings by
evaluating the ability of virus community signatures to classify stool
samples and compared their performance to models built using bacterial
community signatures.

To identify the altered virus communities associated with colorectal
cancer, we built and tested random forest models for classifying stool
samples as belonging to individuals with either cancerous or healthy
colons. We confirmed that our bacterial 16S rRNA gene model replicated
the performance of the original report which used logit models instead
of random forest models \textbf{(Figure \ref{predmodel} A)} (8). We then
compared the bacterial OTU model to a model built using OVU relative
abundances. The viral model performed as well as the bacterial model
(corrected p-value = 0.7), with the viral and bacterial models achieving
mean area under the curve (AUC) values of 0.767 and 0.772, respectively
\textbf{(Figure \ref{predmodel} A - B)}. To evaluate the ability of both
bacterial and viral biomarkers to classify samples, we built a combined
model that used both bacterial and viral community data. The combined
model did not yield a statistically significant performance improvement
beyond the viral (corrected p-value = 0.1) and bacterial (corrected
p-value = 0.2) models, yielding an AUC of 0.804 \textbf{(Figure
\ref{predmodel} A - B)}.

To determine the advantage of viral metagenomic methods over bacterial
metagenomic methods, we compared the viral model to a model built using
OGU relative abundance profiles from bacterial metagenomic shotgun
sequencing data. This model performed worse than the other models (mean
AUC = 0.467) \textbf{(Figure \ref{predmodel} A - B)}. Because the
coverage provided by the metagenomic sequencing was not as deep as the
equivalent 16S rRNA gene sequencing, we attempted to compare the
approaches at a common sequencing depth. This investigation revealed
that the bacterial 16S rRNA gene model was strongly driven by sparse and
low abundance OTUs \textbf{(Figure \ref{16scompare})}. Removal of OTUs
with a median abundance of zero resulted in the removal of six OTUs, and
a loss of model performance down to what was observed in the
metagenome-based model \textbf{(Figure \ref{16scompare} A)}. The
majority of these OTUs had a relative abundance lower than 1\% across
the samples \textbf{(Figure \ref{16scompare} B)}. Although the features
in the viral model also were of low abundance \textbf{(Figure
\ref{threewaymodel} F)}, the coverage was sufficient for high model
performance, likely because viral genomes are orders of magnitude
smaller than bacterial genomes. Thus, the targeted 16S rRNA gene
sequencing approach, which represented only a fraction of the bacterial
metagenomic sequencing depth, was more effective for detecting
colorectal cancer in stool samples. Despite the recent loss of
enthusiasm for 16S rRNA gene sequencing in favor of shotgun metagenomic
techniques, 16S rRNA gene sequencing is still a superior methodological
approach for some important applications.

The association between the bacterial and viral communities and
colorectal cancer was driven by a few important microbes.
\emph{Fusobacterium} was the primary driver of the bacterial association
with colorectal cancer, which is consistent with its previously
described oncogenic potential \textbf{(Figure \ref{predmodel} C)}(22).
The virome signature also was driven by a few OVUs, suggesting a role
for these viruses in tumorigenesis \textbf{(Figure \ref{predmodel} D)}.
The identified viruses were bacteriophages, belonging to
\emph{Siphoviridae}, \emph{Myoviridae}, and ``unclassified'' phage taxa.
Many of the important viruses were unidentifiable (denoted ``unknown'').
This is common in viromes across habitats; studies have reported as much
as 95\% of virus sequences belonging to unknown genomic units (14,
28--30). When the bacterial and viral community signatures were
combined, both bacterial and viral organisms drove the community
association with cancer \textbf{(Figure \ref{predmodel} E)}.

\subsection{Phage Influence Between CRC
Stages}\label{phage-influence-between-crc-stages}

Because previous work has identified shifts in which bacteria were most
important at different stages of colorectal cancer (8, 20, 22), we
explored whether shifts in the relative influence of specific phages
could be detected between healthy, adenomatous, and cancerous colons. We
evaluated community shifts between the two disease stage transitions
(healthy to adenomatous and adenomatous to cancerous) by building random
forest models to compare only the diagnosis groups around the
transitions. While bacterial OTU models performed equally well for all
disease class comparisons, the virome model performances differed
\textbf{(Figure \ref{transitionmodels} A-B)}. Like bacteria
\textbf{(Figure \ref{transitionmodels} F-H)}, different virome members
were important between the healthy to adenomatous and adenomatous to
cancerous stages \textbf{(Figure \ref{transitionmodels} C-E)}.

After evaluating our ability to classify samples between two disease
states, we performed a three-class random forest model including all
disease states. The 16S rRNA gene model yielded a mean AUC of 0.774 and
outperformed the viral community model, which yielded a mean AUC of
0.672 (p-value \textless{} 0.001, \textbf{Figure \ref{threewaymodel}
A-C}). The microbes important for the healthy versus cancer and healthy
versus adenoma models were also important for the three-class model
\textbf{(Figure \ref{threewaymodel} D-E)}. The most important bacterium
in the two and three class models was the same \emph{Fusobacterium} (OTU
4) \textbf{(Figure \ref{predmodel} C, Figure \ref{threewaymodel} D)}.
The viruses most important to the three-class model were identified as
bacteriophages \textbf{(Figure \ref{predmodel} D, Figure
\ref{threewaymodel} E)}, but not all important OVUs were of increased
abundance in the diseased state \textbf{(Figure \ref{threewaymodel} F)}.

\subsection{Phage Dominance in CRC
Virome}\label{phage-dominance-in-crc-virome}

Differences in the colorectal cancer virome could have been driven
directly by eukaryotic viruses or indirectly by bacteriophages acting
through their bacterial hosts. To better understand the types of viruses
that were important for colorectal cancer, we identified the virome OVUs
as being similar to either eukaryotic viruses or bacteriophages. The
most important viruses to the classification model were identified as
bacteriophages (\textbf{Figure \ref{threewaymodel})}. Overall, we were
able to identify 78.8\% of the OVUs as known viruses, and 93.8\% of
those viral OVUs aligned to bacteriophage reference genomes. It is
important to note that this could have been influenced by our
methodological biases against enveloped viruses (more common of
eukaryotic viruses than bacteriophage), due to chloroform and DNase
treatment for purification.

We evaluated whether the phages in the community were primarily lytic
(i.e.~obligately lyse their hosts after replication) or temperate
(i.e.~able to integrate into their host's genome to form a lysogen, and
subsequently transition to a lytic mode). We accomplished this by
identifying three markers for temperate phages in the OVU representative
sequences: 1) presence of phage integrase genes, 2) presence of known
prophage genes, according the the ACLAME (A CLAssification of Mobile
genetic Elements) database, and 3) nucleotide similarity to regions of
bacterial genomes (29, 31, 32). We found that the majority of the phages
were temperate and that the overall fraction of temperate phages
remained consistent throughout the healthy, adenomatous, and cancerous
stages \textbf{(Figure \ref{replicationstyles} E)}. These findings were
consistent with previous reports suggesting the gut virome is primarily
composed of temperate phages (13, 18, 31, 33).

\subsection{Community Context of Influential
Phages}\label{community-context-of-influential-phages}

Because the link between colorectal cancer and the virome was driven by
bacteriophages, we hypothesized that the influential phages were
primarily predators of the influential bacteria, and thus influenced
their relative abundance through predation. If this hypothesis were
true, we would expect a correlation between the relative abundances of
influential bacteria and phages. Instead, we observed a strikingly low
correlation between bacterial and phage relative abundances
\textbf{(Figure \ref{correlations} A,C)}. Overall, there was an absence
of correlation between the most influential OVUs and bacterial OTUs
\textbf{(Figure \ref{correlations} B)}. This evidence supported our null
hypothesis that the influential phages were not primarily predators of
influential bacteria.

Given these findings, we hypothesized that the most influential phages
were acting by infecting a wide range of bacteria in the overall
community, instead of just the influential bacteria. In other words, we
hypothesized that the influential bacteriophages were community hubs
(i.e.~central members) within the bacteria and phage interactive
network. We investigated the potential host ranges of all phage OVUs
using a previously developed random forest model that relies on sequence
features to predict which phages infected which bacteria in the
community \textbf{(Figure \ref{network} A)} ({\textbf{???}}). The
predicted interactions were then used to identify phage community hubs.
We calculated the alpha centrality (i.e.~measure of importance in the
ecological network) of each phage OVU's connection to the rest of the
network. The phages with high centrality values were defined as
community hubs. Next, the centrality of each OVU was compared to its
importance in the colorectal cancer classification model. Phage OVU
centrality was significantly and positively correlated with importance
to the disease model (p-value = 0.02, R = 0.14), suggesting that phages
important in driving colorectal cancer also were more likely to be
community hubs \textbf{(Figure \ref{network} B)}. Together these
findings supported our hypothesis that influential phages were hubs
within their microbial communities and had broad host ranges.

\section{Discussion}\label{discussion}

Because of their propensity for mutagenesis and capacity for modulating
their host functionality, many viruses are oncogenic (1--4). Some
bacteria also have oncogenic properties, suggesting that bacteriophages
may play an indirect role in promoting carcinogenesis by influencing
bacterial community composition and dynamics (8--10). Despite their
carcinogenic potential and the strong association between bacteria and
colorectal cancer, a mechanistic link between virus colorectal
communities and colorectal cancer has yet to be evaluated. Here we show
that, like colonic bacterial communities, the colon virome was altered
in patients with colorectal cancer relative to those with healthy
colons. Our findings support a working hypothesis for oncogenesis by
phage-modulated bacterial community composition.

Using our findings here, we have begun to delineate the role the colonic
virome plays in colorectal cancer in the form of a working hypothesis
that will guide our future studies \textbf{(Figure \ref{modelsummary}
A)}. We found that basic diversity metrics of alpha diversity (richness
and Shannon diversity) and beta diversity (Bray-Curtis dissimilarity)
were insufficient for identifying virome community differences between
healthy and cancerous states. By implementing a more sophisticated
machine learning approach (random forest classification), we detected
strong associations between the colon virus community composition and
colorectal cancer. The colorectal cancer virome was composed primarily
of bacteriophages. These phage communities were not exclusively
predators of the most influential bacteria, as demonstrated by the lack
of correlation between the abundances of the bacterial and phage
populations. Instead, we identified influential phages as being
community hubs, suggesting phages influence cancer by altering the
greater bacterial community instead of directly modulating the
influential bacteria. Our previous work has shown that modifying colon
bacterial communities alters colorectal cancer progression and tumor
burden in mice (10, 20). This provides a precedent for phage indirectly
influencing colorectal cancer progression by altering the bacterial
community composition. Overall, our data support a model in which the
bacteriophage community modulates the bacterial community, and through
those interactions indirectly influences the bacteria driving colorectal
cancer progression \textbf{(Figure \ref{modelsummary} A)}. Although our
evidence suggested phages indirectly influenced colorectal cancer
development, we were not able to rule out the role of phages directly
interacting with the human host (34, 35).

In addition to modeling the potential connections between virus
communities, bacterial communities, and colorectal cancer, we also used
our data and existing knowledge of phage biology to develop a working
hypothesis for the mechanisms by which this may occur. This was done by
incorporating our findings into the current model for colorectal cancer
development \textbf{(Figure \ref{modelsummary} B)} (22). We hypothesize
that the process begins with broadly infectious phages in the colon
lysing and thereby disrupting the existing bacterial communities. This
shift opens novel niche space that enabled opportunistic bacteria (such
as \emph{Fusobacterium nucleatum}) to colonize. Once the initial
influential founder bacteria establish themselves in the epithelium,
secondary opportunistic bacteria are able to adhere to the founders,
colonize, and establish a biofilm. Phages may play a role in biofilm
dispersal and growth by lysing bacteria within the biofilm, a process
important for effective biofilm growth (36). The oncogenic bacteria may
then be able to transform the epithelial cells and disrupt tight
junctions to infiltrate the epithelium, thereby initiating an
inflammatory immune response. As the adenomatous polyps developed and
progressed towards carcinogenesis, we observed a shift in the phages and
bacteria whose relative abundances were most influential. As the
bacteria enter their oncogenic synergy with the epithelium, we
conjecture that the phages continue mediating biofilm dispersal. This
process would thereby support the colonized oncogenic bacteria by lysing
competing cells and releasing nutrients to other bacteria in the form of
cellular lysates. In addition to highlighting the likely mechanisms by
which the colorectal cancer virome is interacting with the bacterial
communities this model will guide future research investigations of the
role the virome plays colorectal cancer.

In addition to the diagnostic ramifications for understanding the
colorectal cancer microbiome, our findings suggest that viruses, while
understudied and currently under-appreciated in the human microbiome,
are likely to be an important contributor to human disease. Viral
community dynamics have the potential to provide an abundance of
information to supplement those of bacterial communities. Evidence has
suggested that the virome is a crucial component to the microbiome and
that bacteriophages are important players. Bacteriophage and bacterial
communities cannot maintain stability and co-evolution without one
another (6, 37). Not only is the human virome an important element to
consider in human health and disease (12--18), but our findings support
that it is likely to have a significant impact on cancer etiology and
progression.

\section{Materials and Methods}\label{materials-and-methods}

\subsection{Analysis Source Code \& Data
Availability}\label{analysis-source-code-data-availability}

All study sequences are available on the NCBI Sequence Read Archive
under the BioProject ID \texttt{PRJNA389927}.

All associated source code is available at the following GitHub
repository:

https://github.com/SchlossLab/Hannigan\_CRCVirome\_mBio\_2017

\subsection{Study Design and Patient
Sampling}\label{study-design-and-patient-sampling}

This study was approved by the University of Michigan Institutional
Review Board and all subjects provided informed consent. Design and
sampling of this sample set have been reported previously (8). Briefly,
whole evacuated stool was collected from patients who were 18 years of
age or older, able to provide informed consent, have had colonoscopy and
histologically confirmed colonic disease status, had not had surgery,
had not had chemotherapy or radiation, and were free of known
co-morbidities including HIV, chronic viral hepatitis, HNPCC, FAP, and
inflammatory bowel disease. Samples were collected from four geographic
locations: Toronto (Ontario, Canada), Boston (Massachusetts, USA),
Houston (Texas, USA), and Ann Arbor (Michigan, USA). Ninety patients
were recruited to the study, thirty of which were designated healthy,
thirty with detected adenomas, and thirty with detected carcinomas.

\subsection{16S rRNA Gene Sequence Data Acquisition \&
Processing}\label{s-rrna-gene-sequence-data-acquisition-processing}

The 16S rRNA gene sequences associated with this study were previously
reported (8). Sequence (fastq) and metadata files were downloaded from:

http://www.mothur.org/MicrobiomeBiomarkerCRC

The 16S rRNA gene sequences were analyzed as described previously,
relying on the mothur software package (v1.37.0) (38, 39). Briefly, the
sequences were de-replicated, aligned to the SILVA database (40),
screened for chimeras using UCHIME (41), and binned into operational
taxonomic units (OTUs) using a 97\% similarity threshold. Abundances
were normalized for uneven sequencing depth by randomly sub-sampling to
10,000 sequences, as previously reported (23).

\subsection{Whole Metagenomic Library Preparation \&
Sequencing}\label{whole-metagenomic-library-preparation-sequencing}

DNA was extracted from stool samples using the PowerSoil-htp 96 Well
Soil DNA Isolation Kit (Mo Bio Laboratories) using an EPMotion 5075
pipetting system. Purified DNA was used to prepare a shotgun sequencing
library using the Illumina Nextera XT library preparation kit according
to the standard kit protocol, including 12 cycles of limited cycle PCR.
The tagmentation time was increased from five minutes to ten minutes to
improve DNA fragment length distribution. The library was sequenced
using one lane of the Illumina HiSeq4000 platform and yielded 125 bp
paired end reads.

\subsection{Virus Metagenomic Library Preparation \&
Sequencing}\label{virus-metagenomic-library-preparation-sequencing}

Genomic DNA was extracted from purified virus-like particles (VLPs) from
stool samples, using a modified version of a previously published
protocol (29, 31, 42, 43). Briefly, an aliquot of stool
(\textasciitilde{}0.1 g) was resuspended in SM buffer (Crystalgen;
Catalog \#: 221-179) and vortexed to facilitate resuspension. The
resuspended stool was centrifuged to remove major particulate debris
then filtered through a 0.22-µm filter to remove smaller contaminants.
The filtered supernatant was treated with chloroform for ten minutes
with gentle shaking, so as to lyse contaminating cells including
bacteria, human, fungi, etc. The exposed genomic DNA from the lysed
cells was degraded by treating the samples with 5U of DNase for one hour
at 37C. DNase was deactivated by incubating the sample at 75C for ten
minutes. The DNA was extracted from the purified virus-like particles
(VLPs) using the Wizard PCR Purification Preparation Kit (Promega).
Disease classes were staggered across purification runs to prevent run
variation as a confounding factor. As for whole community metagenomes,
purified DNA was used to prepare a shotgun sequencing library using the
Illumina Nextera XT preparation kit according to the standard kit
protocol. The tagmentation time was increased from five minutes to ten
minutes to improve DNA fragment length distribution. The PCR cycle
number was increased from twelve to eighteen cycles to address the low
biomass of the samples, as has been described previously (29). The
library was sequenced using one lane of the Illumina HiSeq4000 platform
and yielded 125 bp paired end reads.

\subsection{Metagenome Quality
Control}\label{metagenome-quality-control}

Both the viral and whole community metagenomic sample sets were
subjected to the same quality control procedures. The sequences were
obtained as de-multiplexed fastq files and subjected to 5' and 3'
adapter trimming using the CutAdapt program (v1.9.1) with an error rate
of 0.1 and an overlap of 10 (44). The FastX toolkit (v0.0.14) was used
to quality trim the reads to a minimum length of 75 bp and a minimum
quality score of 30 (45). Reads mapping to the human genome were removed
using the DeconSeq algorithm (v0.4.3) and default parameters (46).

\subsection{Contig Assembly \&
Abundance}\label{contig-assembly-abundance}

Contigs were assembled using paired end read files that were purged of
sequences without a corresponding pair (e.g.~one read removed due to low
quality). The Megahit program (v1.0.6) was used to assemble contigs for
each sample using a minimum contig length of 1000 bp and iterating
assemblies from 21-mers to 101-mers by 20 (47). Contigs from the virus
and whole metagenomic sample sets were concatenated within their
respective groups. Abundance of the contigs within each sample was
calculated by aligning sequences back to the concatenated contig files
using the bowtie2 global aligner (v2.2.1), with a 25 bp seed length and
an allowance of one mismatch (48). Abundance was corrected for contig
reference length and the number of contigs included in each operational
genomic unit. Abundance was also corrected for uneven sampling depth by
randomly sub-sampling virome and whole metagenomes to 1,000,000 and
500,000 reads, respectively, and by removing samples with fewer total
reads than the threshold. Thresholds were set for maximizing sequence
information while minimizing numbers of lost samples.

\subsection{Operational Genomic Unit
Classification}\label{operational-genomic-unit-classification}

Much like operational taxonomic units (OTUs) are used as an operational
definition of similar 16S rRNA gene sequences, we defined closely
related bacterial contig sequences as operational genomic units (OGUs)
and virus contigs as operational viral units (OVUs) in the absence of
taxonomic identity. OGUs and OVUs were defined with the CONCOCT
algorithm (v0.4.0) which bins related contigs by similar tetra-mer and
co-abundance profiles within samples using a variational Bayesian
approach (49). CONCOCT was used with a length threshold of 1000 bp for
virus contigs and 2000 bp for bacteria.

\subsection{Diversity}\label{diversity}

Alpha and beta diversity were calculated using the operational viral
unit abundance profiles for each sample. Sequences were rarefied to
100,000 sequences. Samples with less than the cutoff were removed from
the analysis. Alpha diversity was calculated using the Shannon diversity
and richness metrics. Beta diversity was calculated using the
Bray-Curtis metric (mean of 25 random sub-sampling iterations), and the
statistical significance between the disease state clusters was assessed
using an analysis of similarity (ANOSIM) with a post-hoc multivariate
Tukey test. All diversity calculations were performed in R using the
Vegan package (50).

\subsection{Classification Modeling}\label{classification-modeling}

Classification modeling was performed in R using the Caret package (51).
OTU, OVU, and OGU abundance data was preprocessed by removing features
(OTUs, OVUs, and OGUs) that were present in less than thirty of the
samples. This served both as an effective feature reduction technique
and made the calculations computationally feasible. The binary random
forest model was trained using the Area Under the receiver operating
characteristic Curve (AUC) and the three-class random forest model was
trained using the mean AUC. Both were validated using five-fold nested
cross validation to prevent over-fitting on the tuning paramters. Each
training set was repeated five times, and the model was tuned for mtry
values. For consistency and accurate comparison between feature groups
(e.g., bacteria, viruses), the sample model parameters were used for
each group. The maximum AUC during training was recorded across twenty
iterations of each group model to test the significance of the
differences between feature set performance. Statistical significance
was evaluated using a Wilcoxon test between two categories, or a
pairwise Wilcoxon test with Bonferroni corrected p-values when comparing
more than two categories.

\subsection{Taxonomic Identification of Operational Genomic
Units}\label{taxonomic-identification-of-operational-genomic-units}

Operational viral units (OVUs) were taxonomically identified using a
reference database consisting of all bacteriophage and eukaryotic virus
genomes present in the European Nucleotide Archives. The longest
contiguous sequence in each operational genomic unit was used as a
representative sequence for classification, as described previously
(52). Each representative sequence was aligned to the reference genome
database using the tblastx alignment algorithm (v2.2.27) and a strict
similarity threshold (e-value \textless{} 1e-25) (53). Annotation was
interpreted as phage, eukaryotic virus, or unknown.

\subsection{Ecological Network Analysis \&
Correlations}\label{ecological-network-analysis-correlations}

The ecological network of the bacterial and phage operational genomic
units was constructed and analyzed as previously described
({\textbf{???}}). Briefly, a random forest model was used to predict
interactions between bacterial and phage genomic units, and those
interactions were recorded in a graph database using \emph{neo4j} graph
databasing software (v2.3.1). The degree of phage centrality was
quantified using the alpha centrality metric in the igraph CRAN package.
A Spearman correlation was performed between model importance and phage
centrality scores.

\subsection{Phage Replication Style
Identification}\label{phage-replication-style-identification}

Phage OVU replication mode was predicted using methods described
previously (29, 31, 32). Briefly, we identified temperate OVUs as
representative contigs containing at least one of three genomic markers:
1) phage integrase genes, 2) prophage genes from the ACLAME database, or
3) genomic similarity to bacterial reference genomes. Integrase genes
were identified in phage OVU representative contigs by aligning the
contigs to a reference database of all known phage integrase genes from
the Uniprot database (Uniprot search term: ``organism:phage gene:int NOT
putative''). Prophage genes were identified in the same way, using the
ACLAME set of reference prophage genes. In both cases, the blastx
algorithm was used with an e-value threshold of 10e-5. Representative
contigs were also identified as potential lysogenic phages by having a
high genomic similarity to bacterial genomes. To accomplish this,
representative phage contigs were aligned to the European Nucleotide
Archive bacterial genome reference set using the blastn algorithm
(e-value \textless{} 10e-25).

\section{Funding Information}\label{funding-information}

GD Hannigan was supported in part by the Molecular Mechanisms in
Microbial Pathogenesis Training Program (T32 AI007528). PD Schloss was
supported by funding from the National Institutes of Health
(P30DK034933). MT Ruffin was supported by funding from the National
Institutes of Health (5U01CA86400). The authors declare no competing
interests.

\section{Acknowledgments}\label{acknowledgments}

The authors thank the Schloss lab members for their underlying
contributions, and the Great Lakes-New England Early Detection Research
Network for providing the fecal samples that were used in this study.

\newpage

\section{References}\label{references}

\hypertarget{refs}{}
\hypertarget{ref-Feng:2008kr}{}
1. Feng H, Shuda M, Chang Y, Moore PS. 2008. Clonal integration of a
polyomavirus in human Merkel cell carcinoma. Science 319:1096--1100.

\hypertarget{ref-Shuda:2011gf}{}
2. Shuda M, Kwun HJ, Feng H, Chang Y, Moore PS. 2011. Human Merkel cell
polyomavirus small T antigen is an oncoprotein targeting the 4E-BP1
translation regulator. Journal of Clinical Investigation 121:3623--3634.

\hypertarget{ref-Schiller:2012ba}{}
3. Schiller JT, Castellsagué X, Garland SM. 2012. A review of clinical
trials of human papillomavirus prophylactic vaccines. Vaccine 30 Suppl
5:F123--38.

\hypertarget{ref-Chang:1994up}{}
4. Chang Y, Cesarman E, Pessin MS, Lee F, Culpepper J, Knowles DM, Moore
PS. 1994. Identification of herpesvirus-like DNA sequences in
AIDS-associated Kaposi's sarcoma. Science 266:1865--1869.

\hypertarget{ref-Harcombe:2005fd}{}
5. Harcombe WR, Bull JJ. 2005. Impact of phages on two-species bacterial
communities. Applied and Environmental Microbiology 71:5254--5259.

\hypertarget{ref-RodriguezValera:2009cr}{}
6. Rodriguez-Valera F, Martin-Cuadrado A-B, Rodriguez-Brito B, Pašić L,
Thingstad TF, Rohwer F, Mira A. 2009. Explaining microbial population
genomics through phage predation. Nature Reviews Microbiology
7:828--836.

\hypertarget{ref-Cortez:2014bk}{}
7. Cortez MH, Weitz JS. 2014. Coevolution can reverse predator-prey
cycles. Proceedings of the National Academy of Sciences of the United
States of America 111:7486--7491.

\hypertarget{ref-Zackular:2014fba}{}
8. Zackular JP, Rogers MAM, Ruffin MT, Schloss PD. 2014. The human gut
microbiome as a screening tool for colorectal cancer. Cancer prevention
research (Philadelphia, Pa) 7:1112--1121.

\hypertarget{ref-Garrett:2015fg}{}
9. Garrett WS. 2015. Cancer and the microbiota. Science 348:80--86.

\hypertarget{ref-Baxter:2014hb}{}
10. Baxter NT, Zackular JP, Chen GY, Schloss PD. 2014. Structure of the
gut microbiome following colonization with human feces determines
colonic tumor burden. Microbiome 2:20.

\hypertarget{ref-Arthur:2012kl}{}
11. Arthur JC, Perez-Chanona E, Mühlbauer M, Tomkovich S, Uronis JM, Fan
T-J, Campbell BJ, Abujamel T, Dogan B, Rogers AB, Rhodes JM, Stintzi A,
Simpson KW, Hansen JJ, Keku TO, Fodor AA, Jobin C. 2012. Intestinal
inflammation targets cancer-inducing activity of the microbiota. Science
338:120--123.

\hypertarget{ref-Ly:2014ew}{}
12. Ly M, Abeles SR, Boehm TK, Robles-Sikisaka R, Naidu M,
Santiago-Rodriguez T, Pride DT. 2014. Altered Oral Viral Ecology in
Association with Periodontal Disease. mBio 5:e01133--14--e01133--14.

\hypertarget{ref-Monaco:2016ita}{}
13. Monaco CL, Gootenberg DB, Zhao G, Handley SA, Ghebremichael MS, Lim
ES, Lankowski A, Baldridge MT, Wilen CB, Flagg M, Norman JM, Keller BC,
Luévano JM, Wang D, Boum Y, Martin JN, Hunt PW, Bangsberg DR, Siedner
MJ, Kwon DS, Virgin HW. 2016. Altered Virome and Bacterial Microbiome in
Human Immunodeficiency Virus-Associated Acquired Immunodeficiency
Syndrome. Cell Host and Microbe 19:311--322.

\hypertarget{ref-Willner:2009dq}{}
14. Willner D, Furlan M, Haynes M, Schmieder R, Angly FE, Silva J,
Tammadoni S, Nosrat B, Conrad D, Rohwer F. 2009. Metagenomic analysis of
respiratory tract DNA viral communities in cystic fibrosis and
non-cystic fibrosis individuals. PLOS ONE 4:e7370.

\hypertarget{ref-Abeles:2015dy}{}
15. Abeles SR, Ly M, Santiago-Rodriguez TM, Pride DT. 2015. Effects of
Long Term Antibiotic Therapy on Human Oral and Fecal Viromes. PLOS ONE
10:e0134941.

\hypertarget{ref-Modi:2013fi}{}
16. Modi SR, Lee HH, Spina CS, Collins JJ. 2013. Antibiotic treatment
expands the resistance reservoir and ecological network of the phage
metagenome. Nature 499:219--222.

\hypertarget{ref-SantiagoRodriguez:2015gd}{}
17. Santiago-Rodriguez TM, Ly M, Bonilla N, Pride DT. 2015. The human
urine virome in association with urinary tract infections. Frontiers in
Microbiology 6:14.

\hypertarget{ref-Norman:2015kb}{}
18. Norman JM, Handley SA, Baldridge MT, Droit L, Liu CY, Keller BC,
Kambal A, Monaco CL, Zhao G, Fleshner P, Stappenbeck TS, McGovern DPB,
Keshavarzian A, Mutlu EA, Sauk J, Gevers D, Xavier RJ, Wang D, Parkes M,
Virgin HW. 2015. Disease-specific alterations in the enteric virome in
inflammatory bowel disease. Cell 160:447--460.

\hypertarget{ref-Siegel:2014jo}{}
19. Siegel R, Desantis C, Jemal A. 2014. Colorectal cancer statistics,
2014. CA: a cancer journal for clinicians 64:104--117.

\hypertarget{ref-Zackular:2016en}{}
20. Zackular JP, Baxter NT, Chen GY, Schloss PD. 2016. Manipulation of
the Gut Microbiota Reveals Role in Colon Tumorigenesis. mSphere
1:e00001--15.

\hypertarget{ref-Dejea:2014fz}{}
21. Dejea CM, Wick EC, Hechenbleikner EM, White JR, Mark Welch JL,
Rossetti BJ, Peterson SN, Snesrud EC, Borisy GG, Lazarev M, Stein E,
Vadivelu J, Roslani AC, Malik AA, Wanyiri JW, Goh KL, Thevambiga I, Fu
K, Wan F, Llosa N, Housseau F, Romans K, Wu X, McAllister FM, Wu S,
Vogelstein B, Kinzler KW, Pardoll DM, Sears CL. 2014. Microbiota
organization is a distinct feature of proximal colorectal cancers.
Proceedings of the National Academy of Sciences of the United States of
America 111:18321--18326.

\hypertarget{ref-Flynn:2016iu}{}
22. Flynn KJ, Baxter NT, Schloss PD. 2016. Metabolic and Community
Synergy of Oral Bacteria in Colorectal Cancer. mSphere 1:e00102--16.

\hypertarget{ref-Baxter:2016dja}{}
23. Baxter NT, Ruffin MT, Rogers MAM, Schloss PD. 2016. Microbiota-based
model improves the sensitivity of fecal immunochemical test for
detecting colonic lesions. Genome medicine 8:37.

\hypertarget{ref-Zeller:2014ix}{}
24. Zeller G, Tap J, Voigt AY, Sunagawa S, Kultima JR, Costea PI, Amiot
A, Böhm J, Brunetti F, Habermann N, Hercog R, Koch M, Luciani A, Mende
DR, Schneider MA, Schrotz-King P, Tournigand C, Tran Van Nhieu J, Yamada
T, Zimmermann J, Benes V, Kloor M, Ulrich CM, Knebel Doeberitz M von,
Sobhani I, Bork P. 2014. Potential of fecal microbiota for early-stage
detection of colorectal cancer. Molecular systems biology 10:766--766.

\hypertarget{ref-Fearon:2011go}{}
25. Fearon ER. 2011. Molecular genetics of colorectal cancer. Annual
review of pathology 6:479--507.

\hypertarget{ref-Levin:2008kz}{}
26. Levin B, Lieberman DA, McFarland B, Smith RA, Brooks D, Andrews KS,
Dash C, Giardiello FM, Glick S, Levin TR, Pickhardt P, Rex DK, Thorson
A, Winawer SJ, American Cancer Society Colorectal Cancer Advisory Group,
US Multi-Society Task Force, American College of Radiology Colon Cancer
Committee. 2008. Screening and surveillance for the early detection of
colorectal cancer and adenomatous polyps, 2008: a joint guideline from
the American Cancer Society, the US Multi-Society Task Force on
Colorectal Cancer, and the American College of Radiology., pp. 130--160.
\emph{In} CA: A cancer journal for clinicians. The University of Texas
MD Anderson Cancer Center, Houston, TX, USA. John Wiley \& Sons, Ltd.

\hypertarget{ref-Zauber:2015cr}{}
27. Zauber AG. 2015. The impact of screening on colorectal cancer
mortality and incidence: has it really made a difference? Digestive
diseases and sciences 60:681--691.

\hypertarget{ref-Pedulla:2003tu}{}
28. Pedulla ML, Ford ME, Houtz JM, Karthikeyan T, Wadsworth C, Lewis JA,
Jacobs-Sera D, Falbo J, Gross J, Pannunzio NR, Brucker W, Kumar V,
Kandasamy J, Keenan L, Bardarov S, Kriakov J, Lawrence JG, Jacobs WR,
Hendrix RW, Hatfull GF. 2003. Origins of highly mosaic mycobacteriophage
genomes. Cell 113:171--182.

\hypertarget{ref-Hannigan:2015fz}{}
29. Hannigan GD, Meisel JS, Tyldsley AS, Zheng Q, Hodkinson BP,
SanMiguel AJ, Minot S, Bushman FD, Grice EA. 2015. The Human Skin
Double-Stranded DNA Virome: Topographical and Temporal Diversity,
Genetic Enrichment, and Dynamic Associations with the Host Microbiome.
mBio 6:e01578--15.

\hypertarget{ref-Brum:2015iaa}{}
30. Brum JR, Ignacio-Espinoza JC, Roux S, Doulcier G, Acinas SG, Alberti
A, Chaffron S, Cruaud C, Vargas C de, Gasol JM, Gorsky G, Gregory AC,
Guidi L, Hingamp P, Iudicone D, Not F, Ogata H, Pesant S, Poulos BT,
Schwenck SM, Speich S, Dimier C, Kandels-Lewis S, Picheral M, Searson S,
Tara Oceans Coordinators, Bork P, Bowler C, Sunagawa S, Wincker P,
Karsenti E, Sullivan MB. 2015. Ocean plankton. Patterns and ecological
drivers of ocean viral communities. Science 348:1261498--1261498.

\hypertarget{ref-Minot:2011ez}{}
31. Minot S, Sinha R, Chen J, Li H, Keilbaugh SA, Wu GD, Lewis JD,
Bushman FD. 2011. The human gut virome: Inter-individual variation and
dynamic response to diet. Genome Research 21:1616--1625.

\hypertarget{ref-Hannigan:2017ky}{}
32. Hannigan GD, Zheng Q, Meisel JS, Minot SS, Bushman FD, Grice EA.
2017. Evolutionary and functional implications of hypervariable loci
within the skin virome. PeerJ 5:e2959.

\hypertarget{ref-Reyes:2010cwa}{}
33. Reyes A, Haynes M, Hanson N, Angly FE, Heath AC, Rohwer F, Gordon
JI. 2010. Viruses in the faecal microbiota of monozygotic twins and
their mothers. Nature 466:334--338.

\hypertarget{ref-Lengeling:2013ia}{}
34. Lengeling A, Mahajan A, Gally DL. 2013. Bacteriophages as Pathogens
and Immune Modulators? mBio 4:e00868--13--e00868--13.

\hypertarget{ref-Grski:2012fa}{}
35. G rski A, Mi dzybrodzki R, Borysowski J, D browska K, Wierzbicki P,
Ohams M, Korczak-Kowalska G yna, Olszowska-Zaremba N, usiak-Szelachowska
M, K ak M, Jo czyk E, Kaniuga E, Go a A, Purchla S, Weber-D browska B,
Letkiewicz S awomir, Fortuna W, Szufnarowski K, Pawe czyk Z aw, Rog P, K
osowska D. 2012. Phage as a Modulator of Immune Responses, pp. 41--71.
\emph{In} Bacteriophages, part b. Bacteriophage Laboratory, Ludwik
Hirszfeld Institute of Immunology; Experimental Therapy, Polish Academy
of Sciences, Wrocław, Poland. agorski@ikp.pl; Elsevier.

\hypertarget{ref-Rossmann:2015cj}{}
36. Rossmann FS, Racek T, Wobser D, Puchalka J, Rabener EM, Reiger M,
Hendrickx APA, Diederich A-K, Jung K, Klein C, Huebner J. 2015.
Phage-mediated Dispersal of Biofilm and Distribution of Bacterial
Virulence Genes Is Induced by Quorum Sensing. PLoS Pathogens
11:e1004653--17.

\hypertarget{ref-Brockhurst:2013iq}{}
37. Brockhurst MA, Koskella B. 2013. Experimental coevolution of species
interactions. Trends in ecology \& evolution 28:367--375.

\hypertarget{ref-Kozich:2013db}{}
38. Kozich JJ, Westcott SL, Baxter NT, Highlander SK, Schloss PD. 2013.
Development of a dual-index sequencing strategy and curation pipeline
for analyzing amplicon sequence data on the MiSeq Illumina sequencing
platform. Applied and Environmental Microbiology 79:5112--5120.

\hypertarget{ref-Schloss:2009do}{}
39. Schloss PD, Westcott SL, Ryabin T, Hall JR, Hartmann M, Hollister
EB, Lesniewski RA, Oakley BB, Parks DH, Robinson CJ, Sahl JW, Stres B,
Thallinger GG, Van Horn DJ, Weber CF. 2009. Introducing mothur:
open-source, platform-independent, community-supported software for
describing and comparing microbial communities. Applied and
Environmental Microbiology 75:7537--7541.

\hypertarget{ref-Pruesse:2007jc}{}
40. Pruesse E, Quast C, Knittel K, Fuchs BM, Ludwig W, Peplies J,
Glöckner FO. 2007. SILVA: a comprehensive online resource for quality
checked and aligned ribosomal RNA sequence data compatible with ARB.
Nucleic Acids Research 35:7188--7196.

\hypertarget{ref-Edgar:2011gy}{}
41. Edgar RC, Haas BJ, Clemente JC, Quince C, Knight R. 2011. UCHIME
improves sensitivity and speed of chimera detection. Bioinformatics
27:2194--2200.

\hypertarget{ref-Thurber:2009dn}{}
42. Thurber RV, Haynes M, Breitbart M, Wegley L, Rohwer F. 2009.
Laboratory procedures to generate viral metagenomes. Nature protocols
4:470--483.

\hypertarget{ref-Kleiner:2015kd}{}
43. Kleiner M, Hooper LV, Duerkop BA. 2015. Evaluation of methods to
purify virus-like particles for metagenomic sequencing of intestinal
viromes. BMC Genomics 16:7.

\hypertarget{ref-Martin:2011eu}{}
44. Martin M. 2011. Cutadapt removes adapter sequences from
high-throughput sequencing reads. EMBnetjournal 17:10.

\hypertarget{ref-FASTXToolkit:wr}{}
45. Hannon GJ. 2010. FASTX-Toolkit GNU Affero General Public License.

\hypertarget{ref-Schmieder:2011fo}{}
46. Schmieder R, Edwards R. 2011. Fast identification and removal of
sequence contamination from genomic and metagenomic datasets. PLOS ONE
6:e17288.

\hypertarget{ref-Li:2016kd}{}
47. Li D, Luo R, Liu C-M, Leung C-M, Ting H-F, Sadakane K, Yamashita H,
Lam T-W. 2016. MEGAHIT v1.0: A fast and scalable metagenome assembler
driven by advanced methodologies and community practices. METHODS
102:3--11.

\hypertarget{ref-Langmead:2012jh}{}
48. Langmead B, Salzberg SL. 2012. Fast gapped-read alignment with
Bowtie 2. Nature Methods 9:357--359.

\hypertarget{ref-Alneberg:2014fc}{}
49. Alneberg J, Bjarnason BS aacute ri, Bruijn I de, Schirmer M, Quick
J, Ijaz UZ, Lahti L, Loman NJ, Andersson AF, Quince C. 2014. Binning
metagenomic contigs by coverage and composition. Nature Methods 1--7.

\hypertarget{ref-veganCommunityEco:xdLliqSE}{}
50. Oksanen J, Blanchet FG, Friendly M, Kindt R, Legendre P, McGlinn D,
Minchin PR, OHara RB, Simpson GL, Solymos P, Stevens MHH, Szoecs E,
Wagner H. vegan: Community Ecology Package.

\hypertarget{ref-caretClassificatio:ux5fU2Litux5f1}{}
51. Kuhn M. 2016. caret: Classification and Regression Training. CRAN.

\hypertarget{ref-Guidi:2016kf}{}
52. Guidi L, Chaffron S, Bittner L, Eveillard D, Larhlimi A, Roux S,
Darzi Y, Audic S, Berline L, Brum JR, Coelho LP, Espinoza JCI, Malviya
S, Sunagawa S, Dimier C, Kandels-Lewis S, Picheral M, Poulain J, Searson
S, Tara Oceans Consortium Coordinators, Stemmann L, Not F, Hingamp P,
Speich S, Follows M, Karp-Boss L, Boss E, Ogata H, Pesant S, Weissenbach
J, Wincker P, Acinas SG, Bork P, Vargas C de, Iudicone D, Sullivan MB,
Raes J, Karsenti E, Bowler C, Gorsky G. 2016. Plankton networks driving
carbon export in the oligotrophic ocean. Nature 532:465--470.

\hypertarget{ref-Camacho:2009fc}{}
53. Camacho C, Coulouris G, Avagyan V, Ma N, Papadopoulos J, Bealer K,
Madden TL. 2009. BLAST+: architecture and applications. BMC
Bioinformatics 10:1.

\newpage

\section{Figure Legends}\label{figure-legends}

\begin{figure}[htbp]
\centering
\includegraphics{}
\caption{\emph{Cohort and sample processing outline. Thirty subject
stool samples were collected from healthy, adenoma (pre-cancer), and
carcinoma (cancer) patients. Stool samples were split into two aliquots,
the first of which was used for bacterial sequencing and the second
which was used for virus sequencing. Bacterial sequencing was done using
both 16S rRNA amplicon and whole metagenomic shotgun sequencing
techniques. Virus samples were purified for viruses using filtration and
a combination of chloroform (bacterial lysis) and DNase (exposed genomic
DNA degradation). The resulting encapsulated virus DNA was sequenced
using whole metagenomic shotgun sequencing.} \label{sampleproc}}
\end{figure}

\newpage

\begin{figure}[htbp]
\centering
\includegraphics{}
\caption{\emph{Results from healthy vs cancer classification models
built using virome signatures, bacterial 16S rRNA gene sequence
signatures, whole metagenomic signatures, and a combination of virome
and 16S rRNA gene sequence signatures. A) An example ROC curve for
visualizing the performance of each of the models for classifying stool
as coming from either an individual with a cancerous or healthy colon.
B) Quantification of the AUC variation for each model, and how it
compared to each of the other models based on 15 iterations. A pairwise
Wilcoxon test with a false discovery rate multiple hypothesis correction
demonstrated that all models are significantly different from each other
(p-value \textless{} 0.01). C) Mean decrease in accuracy (measurement of
importance) of each operational taxonomic unit within the 16S rRNA gene
classification model when removed from the classification model. Mean is
represented by a point, and bars represent standard error. D) Mean
decrease in accuracy of each operational virus unit in the virome
classification model. E) Mean decrease in accuracy of each operational
genomic unit and operational taxonomic unit in the model using both 16S
rRNA gene and virome features.}\label{predmodel}}
\end{figure}

\newpage

\begin{figure}[htbp]
\centering
\includegraphics{}
\caption{\emph{Relative abundance correlations between bacterial OTUs
and virome OVUs. A) Pearson correlation coefficient values between all
bacterial OTUs (x-axis) and viral OVUs (y-axis) with blue being
positively correlated and red being negatively correlated. Bar plots
indicate the viral (left) and bacterial (bottom) operational unit
importance in their colorectal cancer classification models, such that
the most important units are in the top left corner. B) Magnification of
the boxed region in panel (A), highlighting the correlation between the
most important bacterial OTUs and virome OVUs. The most important
operational units are in the top left corner of the heatmap, and the
correlation scale is the same as panel (A). C) Histogram quantifying the
frequencies of Pearson correlation coefficients between all bacterial
OTUs and virome OVUs.}\label{correlations}}
\end{figure}

\newpage

\begin{figure}[htbp]
\centering
\includegraphics[width=0.50000\textwidth]{}
\caption{\emph{Lysogenic phage relative abundance in disease states.
Phage OVUs were predicted to be either lytic or lysogenic, and the
relative abundance of lysogenic phages was quantified and represented as
a boxplot. No disease groups were statistically
significant.}\label{replicationstyles}}
\end{figure}

\newpage

\begin{figure}[htbp]
\centering
\includegraphics{}
\caption{\emph{Final model and working hypothesis from this study. A)
Basic model illustrating the connections between the virome, bacterial
communities, and colorectal cancer. B) Working hypothesis of how the
bacteriophage community is associated with colorectal cancer and the
associated bacterial community.}\label{modelsummary}}
\end{figure}

\newpage

\section{Supplemental Figure Legends}\label{supplemental-figure-legends}

\beginsupplement

\begin{figure}[htbp]
\centering
\includegraphics{}
\caption{\emph{Basic Quality Control Metrics. A) VLP genomic DNA yield
from all sequenced samples. Each bar represents a sample which is
grouped and colored by its associated disease group. B) Sequence yield
following quality control including quality score filtering and human
decontamination.}\label{qualcontrol}}
\end{figure}

\newpage

\begin{figure}[htbp]
\centering
\includegraphics{}
\caption{\emph{Length and coverage statistics. A) Heated scatter plot
demonstrating the distribution of contig coverage (number of sequences
mapping to each contig) and contig length for the virus metagenomic
sample set. B) Scatter plot illustrating the distribution of operational
viral unit (OVU) length and sequence coverage for the virus metagenomic
sample set. C) Heated scatter plot demonstrating the distribution of
contig coverage and length for the whole metagenomic sample set. D)
Scatter plot illustrating the distribution of operational genomic unit
(OGU) length and sequence coverage for the whole metagenomic sample
set.}\label{contigqc}}
\end{figure}

\newpage

\begin{figure}[htbp]
\centering
\includegraphics{}
\caption{\emph{Operational genomic unit composition stats. A) Strip
chart demonstrating the length and frequency of contigs within each
operational genomic unit of the virome sample set. The y-axis is the
operational genomic unit identifier, and x-axis is the length of each
contig, and each dot represents a contig found within the specified
operational genomic unit. B) Density plot (analogous to histogram) of
the number of virome operational genomic units containing the specific
number of contigs, as indicated by the x-axis. C-D) Sample plots as
panels C and D, but for the whole metagenomic sample
set.}\label{clustercontigqc}}
\end{figure}

\newpage

\begin{figure}[htbp]
\centering
\includegraphics{}
\caption{\emph{Diversity calculations comparing cancer states of the
colorectal virome, based on relative abundance of operational genomic
units in each sample. A) NMDS ordination of community samples, colored
for cancerous (green), pre-cancerous (red), and healthy (yellow). B)
Differences in means between disease group centroids with 95\%
confidence intervals based on an ANOSIM test with a post hoc
multivariate Tukey test. Comparisons (indicated on y-axis) in which the
intervals cross the zero mean difference line (dashed line) were not
significantly different. C) Shannon diversity and D) richness alpha
diversity quantification comparing pre-cancerous (grey), cancerous
(red), and healthy (tan) states.}\label{betaogu}}
\end{figure}

\newpage

\begin{figure}[htbp]
\centering
\includegraphics{}
\caption{\emph{Beta-diversity comparing disease states and the study
negative controls. Differences in means between disease group centroids
with 95\% confidence intervals based on an ANOSIM test with a post hoc
multivariate Tukey test. Comparisons in which the intervals cross the
zero mean difference line (dashed line) were not significantly
different.}\label{betaogunegative}}
\end{figure}

\newpage

\begin{figure}[htbp]
\centering
\includegraphics{}
\caption{\emph{Comparison of bacterial 16S rRNA classification models
with and without OTUs whose median relative abundance are greater than
zero. A) Classification model performance (measured as area under the
curve) for bacteria models using 16S rRNA data both with and without
filtering of samples whose median was zero. Significance was calculated
using a Wilcoxon rank sum test, and the resulting p-value is shown. The
random area under the curve (0.5) is marked with a dashed line. B)
Relative abundance of the six bacterial OTUs removed when filtered for
OTUs with median relative abundance of zero. OTU relative abundance is
separated by healthy (red) and cancerous (grey) samples. Relative
abundance of 1\% is marked by the dashed line.}\label{16scompare}}
\end{figure}

\newpage

\begin{figure}[htbp]
\centering
\includegraphics{}
\caption{\emph{Transition of colorectal cancer importance through
disease progression. A) Virus and B) 16S rRNA gene model performance
(AUC) when discriminating all binary combinations of disease types. Blue
line represents mean performance from multiple random iterations. C-E)
Top ten important phage OVUs when classifying each combination of
disease state, as measured by the mean decrease in accuracy metric. Mean
is represented by a point, and bars represent standard error. Disease
comparison is specified in the top left corner of each panel. F-H) Top
ten important bacterial 16S rRNA gene OTUs for classifying each disease
state combination.}\label{transitionmodels}}
\end{figure}

\newpage

\begin{figure}[htbp]
\centering
\includegraphics{}
\caption{\emph{ROC curves from A) virome and B) bacterial 16S
three-class random forest models tuned on mean AUC. Each curve
represents the ability of the specified class to be classified against
the other two classes. C) Quantification of the mean AUC variation for
each model based on 10 model iterations. A pairwise Wilcoxon test with a
Bonferroni multiple hypothesis correction demonstrated that the models
are significantly different (alpha = 0.01). D) Mean decrease in accuracy
when virome operational genomic units and E) bacterial 16S OTUs are
removed from the respective three-class classification models. Results
based on 25 iterations. F) Relative abundance of the six most important
virome OVUs in the model, with the most important on the right. Line
indicates abundance mean.}\label{threewaymodel}}
\end{figure}

\newpage

\begin{figure}[htbp]
\centering
\includegraphics{}
\caption{\emph{Community network analysis utilizing predicted
interactions between bacteria and phage operational genomic units. A)
Visualization of the community network for our colorectal cancer cohort.
B) Scatter plot illustrating the correlation between importance (mean
decrease in accuracy) and the degree of centrality for each OVU. A
linear regression line was fit to illustrate the correlation (blue)
which was found to be statistically significantly and weakly correlated
(p-value = 0.0173, R = 0.14).}\label{network}}
\end{figure}

\end{document}
